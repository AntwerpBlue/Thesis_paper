% !TeX root = ../main.tex

\chapter{引言}

\section{研究背景与意义}

\subsection{研究背景}

随着人工智能技术的快速迭代和5G网络的全面覆盖,全球数字内容产业正经历着深刻的范式转变。
根据国家广电总局《中国网络视听发展研究报告(2025)》的最新数据,中国网络视听用户规模
已达10.91亿,其中短视频用户规模突破10.40亿,使用率达93.8\%,连续6年保持网络视听应用细分领域首位。
值得注意的是,短视频用户日均使用时长高达156分钟,显著高于其他互联网应用,反映出其在数字生活中的核心地位。

与此同时,网络直播凭借其"即时在场感"的特性,已成为打破时空限制的"万能窗口"。
截至2024年12月,我国网络直播用户规模达8.33亿,同比增长1737万,占网民总数的75.2\%。
这种快速增长的用户基础带来了巨大的消费潜力——国家统计局数据显示,2024年全国网上零售额达15.52万亿元,同比增长7.2\%,
其中近一半的消费行为直接由短视频或直播内容触发。庞大的用户基数和旺盛的内容创作需求,对肖像视频编辑技术
提出了三大核心要求:
\begin{enumerate}
    \item \textbf{实时性需求}:移动端4K视频的实时编辑已成为行业标配,用户期望编辑延迟控制在200ms以内,以满足直播电商等场景的即时需求;
    \item \textbf{智能化演进}:视频编辑正从工具辅助向创意生成转型。多模态大模型(如GPT-4 Vision)的语义理解能力,使用户可以通过自然语言指令实现复杂编辑,如"将背景转换为未来都市"等高级需求;
    \item \textbf{个性化定制}:尽管生成式人工智能极大降低了创作门槛,但仍面临风格同质化、个性化不足的问题。有创作者反馈AI生成的内容往往"千篇一律",难以满足独特风格的需求。
\end{enumerate}

当前,研究能够解决这些关键问题的AI视频编辑技术已成为行业关注的焦点。以PortraitGen为代表的先进算法已能实现100FPS的高效渲染,
支持文本、图像驱动的多模态编辑;Live2Diff采用单向的时间注意力机制,能够以近乎实时的速度将实时视频流
转换为风格化内容,可以应用于实时直播视频的转绘;阿里通义实验室研发的ChatAnyone能够通过音频输入生成高保真表情和上半身动作视频,
也已经被应用于虚拟网红的批量生产。然而,这些前沿算法在实际应用中仍面临着诸多挑战,一方面现有的编辑算法多面向专业的开发者和研究人员,
缺乏直观的交互设计与低代码的操作流程,难以被普通用户使用;另一方面,研究者难以及时获取终端用户的实际需求,造成技术演进与市场需求的脱节。

这一现状凸显了构建用户友好型视频编辑系统的紧迫性。下一代解决方案需在保持专业级质量的同时,通过技术创新实现交互自然化、操作便捷化、功能多样化,
从而进一步激发AI肖像视频编辑的应用潜力。

\subsection{研究意义}

本研究基于PortraitGen算法,构建了一套完整的智能视频编辑系统。我们设计了直观的用户界面与用户工作流,使
用户无需关心算法细节,即可轻松实现视频在文字、图像等多模态输入下的风格化处理、虚拟合成、背景光重构等功能。
同时,我们搭建了管理员系统,使研究者可以通过用户反馈和操作日志,了解用户的使用情况,从而根据用户的真实需求
优化算法,调整研究方向。进一步地,我们在系统构建时保持系统的可扩展性,为未来的算法升级和功能扩展预留了空间。

本文的主要贡献包括:
\begin{itemize}
    \item 提出面向生产环境的PortraitGen工程化方案,实现算法到产品的完整转化
    \item 设计了可扩展的用户工作流,可以搭载多种算法,实现各种算法的切换
    \item 通过实际应用验证,为行业提供可复制的技术方案
\end{itemize}

\section{论文结构}

本文的组织结构如下:
\begin{itemize}
    \item \textbf{引言}:介绍研究的背景与意义
    \item \textbf{相关工作}:介绍传统的肖像编辑与生成方法,详细讨论PortraitGen技术创新;
    \item \textbf{系统设计}:介绍系统的整体需求、用户工作流设计、系统架构设计;
    \item \textbf{功能实现}:详细描述用户管理、视频编辑功能及管理员功能的实现细节;
    \item \textbf{性能评估}:通过功能测试、性能测试及用户体验评估,验证系统的有效性和稳定性;
    \item \textbf{结论}:总结本文的研究成果,提出系统改进方向和未来工作展望。
\end{itemize}

