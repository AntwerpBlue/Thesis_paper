% !TeX root = ../main.tex

\chapter{引言}

\section{研究背景与意义}

\subsection{研究背景}

随着人工智能技术的快速发展和5G网络的全面普及,全球数字内容产业正在经历前所未有的变革。
根据国家广电总局最新发布的《中国网络视听发展研究报告(2025)》显示,中国网络视听用户规模
已达10.91亿,其中短视频用户规模突破10.40亿,日均使用时长高达156分钟,稳居各类互联网应用之首。
这一庞大的用户基数和旺盛的内容创作需求,对视频编辑技术提出了更高要求,特别是在实时性、
智能化和个性化方面。

当前,AI视频编辑技术已成为行业关注的焦点。以PortraitGen为代表的先进算法已能实现100FPS的高效渲染,
支持文本、图像驱动的多模态编辑;Live2Diff采用单向的时间注意力机制,能够以近乎实时的速度将实时视频流
转换为风格化内容;ChatAnyone等风格化肖像视频生成模型的发展也十分迅速。然而,这些前沿算法在实际应用
中仍面临着诸多挑战,由于缺乏完善的用户交互
设计,限制了其广泛推广和应用,提高了前沿技术的使用门槛;另外,因为没有直接面向用户,研究者往往无法了解
用户的真实需求与体验反馈。

\subsection{研究意义}

本研究基于PortraitGen算法,构建了一套完整的智能视频编辑系统。我们设计了直观的用户界面与用户工作流,使
用户无需关心算法细节,即可轻松实现视频在文字、图像等多模态输入下的风格化处理、虚拟合成、背景光重构等功能。
同时,我们搭建了管理员系统,使研究者可以通过用户反馈和操作日志,了解用户的使用情况,从而根据用户的真实需求
优化算法,调整研究方向。

本文的主要贡献包括:
\begin{itemize}
    \item 提出面向生产环境的PortraitGen工程化方案,实现算法到产品的完整转化
    \item 设计了可扩展的用户工作流,可以搭载多种算法,实现各种算法的切换
    \item 通过实际应用验证,为行业提供可复制的技术方案
\end{itemize}

\section{论文结构}

本文的组织结构如下:
\begin{itemize}
    \item \textbf{相关工作}:详细讨论PortraitGen技术原理、现有视频编辑App的现状以及技术栈选择;
    \item \textbf{系统设计}:介绍系统的整体架构、前端设计、后端设计、数据库设计及安全性设计;
    \item \textbf{功能实现}:详细描述用户管理、视频编辑功能及管理员功能的实现细节;
    \item \textbf{性能评估}:通过功能测试、性能测试及用户体验评估,验证系统的有效性和稳定性;
    \item \textbf{改进与展望}:提出系统改进方向和未来工作展望;
    \item \textbf{结论}:总结本文的研究成果,强调本文的贡献及未来展望。
\end{itemize}

