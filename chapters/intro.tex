% !TeX root = ../main.tex

\chapter{引言}

\section{研究背景与意义}

\subsection{研究背景}

随着人工智能技术的快速发展和5G网络的全面普及,全球数字内容产业正在经历前所未有的变革。
根据国家广电总局最新发布的《中国网络视听发展研究报告(2025)》显示,中国网络视听用户规模
已达10.91亿,其中短视频用户规模突破10.40亿,使用率达93.8\%,连续6年保持网络视听应用细分领域第一,
日均使用时长高达156分钟,稳居各类互联网应用之首。网络直播凭借“即时在场感”打破空间限制,成为用户可随身携带的“万能窗口”。
截至去年12月,我国网络直播用户规模为8.33亿,同比增长1737万,占网民总数的75.2\%。

持续增长的短视频与直播用户规模也带来了极大的消费潜力,国家统计局数据显示,去年全国网上零售额达155225亿元,同比增长7.2\%。
调查显示,近一半受访者曾因观看短视频或直播而发生消费行为。庞大的用户基数和旺盛的内容创作需求,对肖像视频编辑技术
提出了三大核心要求:
\begin{enumerate}
    \item \textbf{实时性需求}:用户期望在移动端实现高分辨率视频的实时编辑,以减少等待时间,满足直播业务的实时需求;
    \item \textbf{智能化演进}:视频编辑技术正从“工具辅助”向“创意生成”转变,得益于多模态大模型(如GPT-4 Vision)带来的语义理解能力提升,用户期望通过更为自然、直观的方式与AI交互,实现内容创作;
    \item \textbf{个性化定制}:根据内容创作日益个性化的需求,创作者们期望能够生成独特的风格以满足特定场景和用户需求。
\end{enumerate}

当前,AI视频编辑技术已成为行业关注的焦点。以PortraitGen为代表的先进算法已能实现100FPS的高效渲染,
支持文本、图像驱动的多模态编辑;Live2Diff采用单向的时间注意力机制,能够以近乎实时的速度将实时视频流
转换为风格化内容;ChatAnyone等风格化肖像视频生成模型的发展也十分迅速。然而,这些前沿算法在实际应用
中仍面临着诸多挑战,由于缺乏完善的用户交互
设计,限制了其广泛推广和应用,提高了前沿技术的使用门槛;另外,因为没有直接面向用户,研究者往往无法了解
用户的真实需求与体验反馈。

\subsection{研究意义}

本研究基于PortraitGen算法,构建了一套完整的智能视频编辑系统。我们设计了直观的用户界面与用户工作流,使
用户无需关心算法细节,即可轻松实现视频在文字、图像等多模态输入下的风格化处理、虚拟合成、背景光重构等功能。
同时,我们搭建了管理员系统,使研究者可以通过用户反馈和操作日志,了解用户的使用情况,从而根据用户的真实需求
优化算法,调整研究方向。

本文的主要贡献包括:
\begin{itemize}
    \item 提出面向生产环境的PortraitGen工程化方案,实现算法到产品的完整转化
    \item 设计了可扩展的用户工作流,可以搭载多种算法,实现各种算法的切换
    \item 通过实际应用验证,为行业提供可复制的技术方案
\end{itemize}

\section{论文结构}

本文的组织结构如下:
\begin{itemize}
    \item \textbf{引言}:介绍研究的背景与意义
    \item \textbf{相关工作}:介绍传统的肖像编辑与生成方法,详细讨论PortraitGen技术创新;
    \item \textbf{系统设计}:介绍系统的整体需求、用户工作流设计、系统架构设计;
    \item \textbf{功能实现}:详细描述用户管理、视频编辑功能及管理员功能的实现细节;
    \item \textbf{性能评估}:通过功能测试、性能测试及用户体验评估,验证系统的有效性和稳定性;
    \item \textbf{结论}:总结本文的研究成果,提出系统改进方向和未来工作展望。
\end{itemize}

