
\chapter{相关工作}

\section{PortraitGen技术原理分析}

PortraitGen作为当前先进的肖像视频编辑技术,其核心创新点体现在以下三个关键方面。

\subsection{3D高斯场表示}

传统的视频编辑利用生成对抗网络(GAN)、去噪扩散模型等技术进行内容生成,但这些编辑技术
都无法保证视频帧之间的一致性;为了结果的连续性,许多工作提出了引入包括密集对应、帧间注意力
机制等方法,但由于缺乏对3D场景的理解与3D对象的先验认知,这些模型也没有办法保证视频的一致性。

在PortraitGen中,我们将视频编辑任务从2D提升至3D,并采用3D高斯溅射来表示场景。具体而言,
我们用数百万个3D高斯椭球来表示场景,对于每个以$\symbf{x}_0$为球心的高斯分布,我们有
\begin{equation}
    g(\symbf{x})=\exp{(\symbf{x}-\symbf{x}_0)^T\symbf{\Sigma}(\symbf{x}-\symbf{x}_0)}
\end{equation}
其中$\symbf{\Sigma}$为协方差矩阵,通过旋转和缩放控制椭球的形状,有分解
\begin{equation}
    \symbf{\Sigma}=\symbf{R}\symbf{\Lambda}\symbf{\Lambda}^T\symbf{R}^T
\end{equation}
其中$\symbf{R}$为旋转矩阵,$\symbf{\Lambda}$为缩放矩阵。为了将三维的场景映射到二维平面,
我们还需要定义每个高斯椭球的不透明度$\alpha$和球谐函数,以控制光照和颜色信息。

3D高斯场采用了可微分的溅射渲染,通过将3D高斯投影到2D屏幕空间,我们可以计算每个像素的颜色
\begin{equation}
    C(\symbf{p})=\sum_{i\in N}c_i\alpha_i\prod_{j=1}^{i-1}(1-\alpha_j)
\end{equation}
其中$c_i$为第$i$个高斯的颜色,基于视角可以由球谐函数决定。为了优化由稀疏点云重建的3D高斯场,
我们将给定视角下的渲染结果与真实结果进行比对,定义损失函数为
\begin{equation}
    \mathcal{L}=(1-\lambda)\mathcal{L}_1+\lambda\mathcal{L}_\text{D\_SSIM}
\end{equation}
其中$\mathcal{L}_1$为渲染的像素差异,描述了渲染结果与真是结果的光度差异;$\mathcal{L}_\text{D\_SSIM}$
为