
\chapter{相关工作}

\section{PortraitGen技术原理分析}

PortraitGen作为当前先进的肖像视频编辑技术,其核心创新点体现在以下三个关键方面。

\subsection{3D高斯场表示}

传统的视频编辑利用生成对抗网络(GAN)、去噪扩散模型等技术进行内容生成,但这些编辑技术
都无法保证视频帧之间的一致性;为了结果的连续性,许多工作提出了引入包括密集对应、帧间注意力
机制等方法,但由于缺乏对3D场景的理解与3D对象的先验认知,这些模型也没有办法保证视频的一致性。

在PortraitGen中,我们将视频编辑任务从2D提升至3D,并采用3D高斯溅射来表示场景。具体而言,
我们用数百万个3D高斯椭球来表示场景,对于每个以$\symbf{x}_0$为球心的高斯分布,我们有
\begin{equation}
    g(\symbf{x})=\exp{(\symbf{x}-\symbf{x}_0)^T\symbf{\Sigma}(\symbf{x}-\symbf{x}_0)}
\end{equation}
其中$\symbf{\Sigma}$为协方差矩阵,通过旋转和缩放控制椭球的形状。