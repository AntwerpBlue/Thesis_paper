\chapter{系统实现}

\section{技术栈选择}

\subsection{前端技术}

本系统采用Next.js作为前端框架,主要基于以下考量:
\begin{enumerate}
    \item \textbf{服务器端渲染}:在服务器渲染React组件,以提升用户的首屏加载速度;
    \item \textbf{静态生成}:在构建时生成静态HTML文件,从而提前生成页面,用户访问时即时加载;
    \item \textbf{TypeScript支持}:支持TypeScript,增加类型检查与面向对象。
\end{enumerate}

\subsection{后端技术}

后端采用用python编写的Flask框架,基于以下的开发需求:
\begin{enumerate}
    \item \textbf{轻量级}:作为轻量级Web框架,Flask不强制使用任何特定库,可以根据需求灵活扩展;
    \item \textbf{RESTful API}:Flask采用Werkzeug路由引擎,支持动态URL规则,方便定义API节点,适合开发RESTful API;
    \item \textbf{异步任务处理支持}:Flask支持Redis作为任务队列,方便处理异步任务;
    \item \textbf{算法适配}:使用python开发使其与各种AI算法适配,方便调用处理算法与后续算法扩展。
\end{enumerate}

\subsection{数据管理}

本系统采用MySQL作为数据库,主要基于以下考量:
\begin{enumerate}
    \item \textbf{事务支持}:MySQL保证事务的原子性、一致性、隔离性和持久性,适合处理复杂的数据操作;
    \item \textbf{SQL查询性能}:MySQL的SQL查询性能优秀,足以满足系统的数据库操作要求;
    \item \textbf{关系型数据库}:MySQL是一种关系型数据库,适合存储结构化数据,如用户信息、视频信息、请求信息等。
\end{enumerate}

另外我们使用redis管理任务队列,作为高性能的键值对存储数据库,适合存储任务信息,如任务ID、任务状态、任务进度等,并且
由于其支持发布/订阅模式,方便实现任务队列的异步处理。