% !TeX root = ../main.tex

\ustcsetup{
  keywords  = {学位论文, 摘要, 关键词},
  keywords* = {Dissertation, Abstract, Keywords},
}

\begin{abstract}
  随着数字媒体技术的飞速发展和短视频平台的普及,视频内容创作已成为互联网时代的重要表达方式。
  其中,肖像视频编辑作为视频处理领域的核心应用之一,在影视制作、社交媒体、虚拟现实等领域具有广泛需求。
  传统的生成式人工智能视频编辑方法主要依赖于2D图像处理技术,难以满足3D场景下的高质量、高效率编辑需求。

  在这一背景下,PortraitGen系统代表了当前肖像视频编辑技术的前沿水平。该系统通过将2D肖像视频编辑问题
  提升至3D领域,利用动态3D高斯场确保空间和时间一致性,并创新性地引入神经高斯纹理机制,实现了高质量、
  高效率的多模态肖像视频编辑。
  
  本文基于PortraitGen算法设计并实现了一个完整的视频编辑系统,采用前后端分离的系统架构,开发了完整的
  用户工作流,并实现了管理员监控系统,以支持用户的行为分析和数据监控。

\end{abstract}

\begin{abstract*}
  With the rapid development of digital media technology and the widespread adoption of short 
  video platforms, video content creation has become an essential mode of expression in the 
  internet era. Among these, portrait video editing, as one of the core applications in video 
  processing, holds extensive demand in fields such as film production, social media, and virtual 
  reality. Traditional AIGC video editing methods primarily rely on 2D image 
  processing techniques, making it difficult to meet the requirements for high-quality and 
  high-efficiency editing in 3D scenarios.

  Against this backdrop, the PortraitGen system represents the cutting edge of current portrait 
  video editing technology. By elevating the 2D portrait video editing problem into the 3D domain, 
  the system employs a dynamic 3D Gaussian field to ensure spatial and temporal consistency while 
  innovatively introducing a Neural Gaussian Texture mechanism, thereby achieving high-quality, 
  efficient, and multimodal portrait video editing.

  This paper designs and implements a comprehensive video editing system based on the PortraitGen 
  algorithm. Adopting a frontend-backend decoupled architecture, the system develops a complete 
  user workflow and implements an administrator monitoring system to support user behavior 
  analysis and data monitoring.
\end{abstract*}
