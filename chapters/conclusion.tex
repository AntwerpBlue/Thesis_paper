\chapter{结论}

\section{总结}

本文基于PortraitGen算法,利用Next.js与Flask等技术,实现了一个多模态的肖像视频编辑系统。
该系统支持用户上传视频并通过多种方式进行肖像视频编辑,并实现了用户管理与数据统计等功能。
我们分析了包括功能性、非功能性在内的等多种需求,设计了系统的总体架构与详细结构,利用多种工具
编码实现了系统的各项功能,并通过功能测试验证了系统的正确性与可靠性,详细展示了该编辑系统从
抽象概念到具体实现的全过程。我们将系统的项目代码及相关文档开源在\url{https://github.com/AntwerpBlue/Portrait-Editor},
以供参考。

本文提出的多模态肖像视频编辑系统为解决先进多模态算法难以直接被非专业用户使用的问题提供了一种有效的
解决方案,为多模态AI技术的推广与应用提供了参考。本文提出的系统也为研究者难以有效获得用户需求与用户反馈
的问题提供了一种解决思路,为研究者的需求收集提供了数据来源。

\section{展望}

本文提出的肖像编辑系统实现了PortraitGen算法的图形化界面编辑,能够作为算法研究的展示工具。
但是由于对相关的工具使用不够熟练,以及缺乏系统开发的经验,本文实现的系统仍有一些不足:
\begin{enumerate}
    \item 本系统的上传工作流根据PortraitGen的需求定制,无法通过简单的方式将上传模块重新组合以支持其他算法。
    但是考虑到上传工作流逻辑简单,可以将其封装成模块,根据自定义逻辑重新组合,并在后端上传新的算法以获取新算法支持。
    \item 本系统编码过程中没有设计统一的数据接口,导致前后端的数据交互不够规范,在前端与后端交互时需要额外处理数据格式。
    在将来的开发中需要设计统一数据接口以增加系统的可维护性。
\end{enumerate}